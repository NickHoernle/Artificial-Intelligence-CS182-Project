\documentclass[11pt]{article}
\usepackage{fullpage,amsmath,amsfonts,mathpazo,microtype,nicefrac,graphicx,verbatimbox,listings,hyperref,enumitem,booktabs,mathtools,amssymb,float,subcaption}

\title{
\vspace{1cm}
\textmd{\textbf{CS182 Project Proposal: Some title}}\\
% \normalsize\vspace{0.1in}\small{Due\ on\ \hmwkDueDate}\\
% \vspace{0.1in}\large{\textit{\hmwkClassInstructor\ \hmwkClassTime}}
}

\author{\textbf{Sophie Hilgard \textit{and} Nick Hoernle}}
\date{\today} % Insert date here if you want it to appear below your name

%----------------------------------------------------------------------------------------

\begin{document}

%http://link.springer.com/chapter/10.1007%2F978-3-642-35527-1_44
%https://www.quora.com/Can-machine-learning-predict-stock-prices
%http://file.scirp.org/pdf/SN_2015070917142293.pdf
%http://journals.plos.org/plosone/article?id=10.1371/journal.pone.0138441

\maketitle

\lstset{language=Python, basicstyle=\footnotesize} % set language as python

\section*{Problem Statement}
Take three people in three locations, each with a different mode of transport available. Find a meeting location which is equidistant (in terms of commute time) for each person and which isn’t just in the middle of nowhere and return the optimum safe path (bike route, car route, bus route) for each person. 

\section{Identification of Specific Related Course Topics}
Algorithms and techniques:
\begin{itemize}
\item Safety-Optimised Bike Routing - i.e. optimise bike paths by factoring in locations of frequent bike accidents, street lighting, bike routes vs main roads, road construction
\item One possible approach for this could be graph search with path lengths calculated as path length multiplied by the frequency of accidents i.e. penalising paths where accidents frequently occur. For road segments where no accidents occur the length would be equal to the actual length. With this approach the algorithm should find the shortest path but taking into account the cost of roads which have a high frequency of accidents. The path found with this approach could be compared to the shortest path without accident-weighting to see what the difference is in terms of path length.
\item Safety-Optimised walking paths - a similar approach to the above could be used to find safest walking paths using crime data and street lighting information.  
\item Equidistant commute times - CSP?
\item Modes of transport - CSP?
\end{itemize}

Extensions:
Suggestions for places to meet at the meeting location 


\section{Examples of Expected Behavior of the System}

\section{Problems that we may encounter}

\section{List of Papers or other Resources}
Including Accident Information in Automatic Bicycle Route Planning for Urban Areas https://www.hindawi.com/journals/usr/2011/362817/


Data sources:
MBTA Schedules and Trip Planning Data API http://www.mbta.com/rider\_tools/developers/default.asp?id=21895
City of Cambridge GIS data GEOJSON files http://cambridgegis.github.io/gisdata.html
Contains data on infrastructure, landmarks ets
Cambridge Open Data:
\begin{itemize}
\item Accidents - list of crashes involving motor vehicles, bicycles and/or pedestrians reported in the City of Cambridge from January 2010 through June 2016 https://data.cambridgema.gov/Public-Safety/Crashes/ybny-g9cv/data
\item Crime - list of crime incidents featured in the Cambridge Police Department’s Annual Crime Reports and reported in the City of Cambridge from 2009-2016. Includes more than 40 types of crimes. typeshttps://data.cambridgema.gov/Public-Safety/Crime-Reports/xuad-73uj
\item Metered Parking Spaces - https://data.cambridgema.gov/Traffic-Parking-and-Transportation/Metered-Parking-Spaces/6h7q-rwhf
\item Boston Open data https://data.cityofboston.gov/
\end{itemize}

Example proposals from previous years
http://isites.harvard.edu/fs/docs/icb.topic707165.files/pdfs/hysen\_noronha.pdf
http://isites.harvard.edu/fs/docs/icb.topic914900.files/pdfs/proposal-hysen-noronha.pdf



\end{document}