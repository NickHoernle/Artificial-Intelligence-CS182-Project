\documentclass[11pt]{article}
\usepackage{fullpage,amsmath,amsfonts,mathpazo,microtype,nicefrac,graphicx,verbatimbox,listings,hyperref,enumitem,booktabs,mathtools,amssymb,float,subcaption}

\title{
\vspace{1cm}
\textmd{\textbf{CS182 Project Proposal: Some title}}\\
% \normalsize\vspace{0.1in}\small{Due\ on\ \hmwkDueDate}\\
% \vspace{0.1in}\large{\textit{\hmwkClassInstructor\ \hmwkClassTime}}
}

\author{\textbf{Sophie Hilgard \textit{and} Nick Hoernle}}
\date{\today} % Insert date here if you want it to appear below your name

%----------------------------------------------------------------------------------------

\begin{document}

%http://link.springer.com/chapter/10.1007%2F978-3-642-35527-1_44
%https://www.quora.com/Can-machine-learning-predict-stock-prices
%http://file.scirp.org/pdf/SN_2015070917142293.pdf
%http://journals.plos.org/plosone/article?id=10.1371/journal.pone.0138441

\maketitle

\lstset{language=Python, basicstyle=\footnotesize} % set language as python

\section*{Problem Statement}
The driving force behind a stock's price consists of a vast interaction of factors that may include primary factors such as the sales, revenue and costs of the company. However, external forces such as the market condition, consumer satisfaction with the product or service and consumer herding (e.g. in a bare market many investors sell options due to a bare mindset in the market). The price variability of the stock is thus not inherently random and thus can be modeled using a Hidden Markov Model (HMM) where the stock price at close of business is dependent upon the price leading up to that day and additional evidence that can be found from corresponding sources.\\
\par We propose that using Social Media feeds such as Twitter can enable us to design a Hidden Markov Model to add useful evidence into the model when making a prediction about a stock's price. The primary aim of this project will therefore be to model the system as an HMM, explore the Twitter API for simple tweets that correspond to a certain business, use off the shelf python-based tools to categorise the twitter data based on tweet sentiment and finally use this sentiment data in refining the HMM that is used to predict the stock price.

\section{Identification of Specific Related Course Topics}

\section{Examples of Expected Behavior of the System}

\section{Problems that we may encounter}

\section{List of Papers or other Resources}
Stock price database. We need to see if there is any access to bloomberg or something like that.


\end{document}